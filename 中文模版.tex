\documentclass{article}

\usepackage{geometry}
\geometry{a4paper}
%\usepackage[UTF8, heading = false, scheme = plain]{ctex}%格式
\usepackage{ctex}
%\usepackage{authblk} %添加机构,需要安装preprint包
\usepackage{graphicx} %添加图片
\usepackage{amsthm}
\usepackage{amsmath}
\renewcommand{\vec}[1]{\boldsymbol{#1}} % 生产粗体向量,而不是带箭头的向量
\usepackage{amssymb}
\usepackage{booktabs} % excel导出的大表格

%\newtheorem{definition}{Definition} %英文
%\newtheorem{theorem}{Theorem}
\newtheorem{definition}{定义} %中文
\newtheorem{lemma}{引理}
\newtheorem{theorem}{定理}
%\newenvironment{proof}{{\noindent\it 证明}\quad}{\hfill $\square$\par}

\DeclareMathOperator{\Ima}{Im}%定义新符号
\DeclareMathOperator{\Rank}{rank}%定义求秩算子

\title{数学-计算机视觉}
\author{Jinzhong Xu}
%\affil{中国科学院}

%date{2019年10月30日} %注释后显示为编译时日期

\begin{document}
\maketitle

\tableofcontents
\newpage
% 生成目录,请删除上面两行注释

%\listoffigures
%\newpage
% 生成图片列表,请删除上面两行注释

\section{数学}

\subsection{高等代数与最优化理论}

%\begin{figure}[htbp] %htbp
%\centering
%\includegraphics[scale=0.6]{gradient.png}
%\caption{this is a figure demo}
%\label{fig:label}
%\end{figure}

$$
A =
\begin{pmatrix}
	1 & 2 & 3 \\
	2 & 3 & 1 \\
	3 & 1 & 2
\end{pmatrix}
$$
$\rho(A) = 6$.

\subsection{数学分析}

\begin{equation}\label{equ:sin}
\sin(z) = 2
\end{equation}
其中,$z\in \mathbb{C}$.

您能求解方程(\ref{equ:sin})吗?

\section{计算机视觉}

\subsection[Panoptic Segmentation]{全景分割}

\subsubsection{语义分割}

\subsubsection{实例分割}

\subsection[3D Reconstruction]{三维重建}

\end{document}